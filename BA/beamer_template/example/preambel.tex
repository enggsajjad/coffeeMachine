
%%
%% RWTH Beamer Theme (layout based on RWTH PowerPoint template)
%% by Georg Wassen, Lehrstuhl für Betriebssysteme, 2013
%%    wassen@lfbs.rwth-aachen.de
%% with modifications by Gerrit Toehgiono, Mentoring Informatik, 2016-2018
%% to match the "current old" RWTH template
%%
%% The templates are derived from the beamer documentation and the provided templates,
%% hence, the same licence applies:
%%
%% ----------------------------------------------------------------
%% |  This file may be distributed and/or modified                |
%% |                                                              |
%% |  1. under the LaTeX Project Public License and/or            |
%% |  2. under the GNU Public License.                            |
%% |                                                              |
%% |  See the file doc/licenses/LICENSE for more details.         |
%% ----------------------------------------------------------------
%%
%% Version 0.1    20.12.2012    Initial presentation using this theme
%% Version 0.2    08.01.2013    Extracted layout and some example slides
%% Version 0.3    12.01.2013    Improved handling of \subsection (used as frame title),
%%                              Support institute's logo
%% Version 0.4    22.01.2013    Improved spacing between lines in header & footer
%% Version 0.5    01.02.2013    No right sidebar: set width correctly
%% Version 1.0    06.03.2014    Initial Tag on GitHub
%%
%% Version 1.1    31.03.2018    Modifications for CS Mentoring
%%
%% TODO
%%  - copyright of slides?
%%  - count slides with same subsection (needs probably overloading \subsection) and print: 1/N
%%    (needs to count the number of frames in each subsection)

%%%%%%%%%%%%%%%%%%%%%%%%%%%%%%%%%%%%%
%% Select input file encoding:
%%   utf8   - UTF-8, nowadays standard on most operating sytems
%%   latin1 - ISO-8859-1
\usepackage[utf8]{inputenc}

%%%%%%%%%%%%%%%%%%%%%%%%%%%%%%%%%%%%%
%% Select language
%%
\usepackage[ngerman]{babel}        % Deutsch, neue Rechtschreibung
%\usepackage[english]{babel}       % English

\usetheme{KIT}
\usepackage[T1]{fontenc}           % Font encoding (don't change!)
\usepackage{lmodern}               % Select Linux Modern Fonts (don't change)
\usepackage{sansmathfonts}         % Sans fonts in math environments
\usepackage{textcomp}              % fix 'missing font symbols' warning
\renewcommand{\rmdefault}{phv}     % Arial like (Helvetica)
\renewcommand{\sfdefault}{phv}     % Arial like (Helvetica)

%% graphics related packages
\usepackage{graphicx}              % needed to include graphics (don't change)
\usepackage[encoding,filenameencoding=utf8]{grffile} % allow utf8 file names in graphics

%%%%%%%%%%%%%%%%%%%%%%%%%%%%%%%%%%%%%
%% import packages for content
%%
\usepackage{listings}                           % for lstlisting and \lstinline|..|
%% TikZ can be used to /program/ graphics.

%% some TikZ-libraries and settings for the examples...
\usetikzlibrary{shadings}           % GW: color gradients
\usetikzlibrary{arrows,calc,positioning,fit,matrix,shadows,chains,arrows,shapes,spy,fadings}
\usetikzlibrary{pgfplots.units,shapes.symbols,shapes.arrows}
\usetikzlibrary{knots,hobby, decorations.pathreplacing} % für Knotendiagramme
\include{my_includes}

%für Notizen
\usepackage{pgfpages}
%\usepackage{pdfpc-commands} %für Videos
\usepackage{multimedia} % für Videos


%\usepackage{ngerman}
\usepackage{amsmath}
%\usepackage{amsthm}
\usepackage{amssymb}
\usepackage{mathtools}
\usepackage{stmaryrd}
\usepackage{faktor} % für Quotientenräume

\usepackage{subcaption} %für Subfigure
\usepackage{xparse} % für NewDocumentCommand



%\usepackage[german]{todonotes}
%\usepackage[disable]{todonotes}
%%% todo mit Liste, da todo-notes liste nicht mit beamer kompatibel
\usepackage{tcolorbox}

%\newcounter{todo}
\newtcbox{\mytodobox}{colback=orange,colframe=orange!75!black}

\newcommand\todo[1]{%
    %\refstepcounter{todo}
	%\mytodobox{\hypertarget{todo\thetodo}{#1}}
	\mytodobox{#1}
    %\addcontentsline{tod}{subsection}{\protect\hyperlink{todo\thetodo}{\thetodo~#1}\par}
}%

\makeatletter
\newcommand\listoftodos{%
    \@starttoc{tod}}
\makeatother
%%% Todo

\usepackage{import}
\usepackage{autonum}
\usepackage{ifthen}

\usepackage{cleveref}
\crefname{section}{Abschnitt}{Abschnitte}
\Crefname{section}{Abschnitt}{Abschnitte}

\crefname{subsection}{Unterabschnitt}{Unterabschnitte}
\Crefname{subsection}{Unterabschnitt}{Unterabschnitte}

\crefname{Def}{Definition}{Definitionen}
\Crefname{Def}{Definition}{Definitionen}
\crefname{Def_noBreak}{Definition}{Definitionen}
\Crefname{Def_noBreak}{Definition}{Definitionen}

\crefname{Bsp}{Beispiel}{Beispiele}
\Crefname{Bsp}{Beispiel}{Beispiele}
\crefname{Bsp_noBreak}{Beispiel}{Beispiele}
\Crefname{Bsp_noBreak}{Beispiel}{Beispiele}

\crefname{Bem}{Bemerkung}{Bemerkungen}
\Crefname{Bem}{Bemerkung}{Bemerkungen}
\crefname{Bem_noBreak}{Bemerkung}{Bemerkungen}
\Crefname{Bem_noBreak}{Bemerkung}{Bemerkungen}

\crefname{Sa}{Satz}{Sätze}
\Crefname{Sa}{Satz}{Sätze}
\crefname{Sa_noBreak}{Satz}{Sätze}
\Crefname{Sa_noBreak}{Satz}{Sätze}

\crefname{Lem}{Lemma}{Lemmata}
\Crefname{Lem}{Lemma}{Lemmata}
\crefname{Lem_noBreak}{Lemma}{Lemmata}
\Crefname{Lem_noBreak}{Lemma}{Lemmata}

\crefname{Ko}{Korollar}{Korollare}
\Crefname{Ko}{Korollar}{Korollare}
\crefname{Ko_noBreak}{Korollar}{Korollare}
\Crefname{Ko_noBreak}{Korollar}{Korollare}

\crefname{Erg}{Ergänzung}{Ergänzungen}
\Crefname{Erg}{Ergänzung}{Ergänzungen}
\crefname{Erg_noBreak}{Ergänzung}{Ergänzungen}
\Crefname{Erg_noBreak}{Ergänzung}{Ergänzungen}

\crefname{Bed}{Bedingung}{Bedingungen}
\Crefname{Bed}{Bedingung}{Bedingungen}
\crefformat{Bed}{Bedingung~(#2#1#3)}
\Crefformat{Bed}{Bedingung~(#2#1#3)}
\crefmultiformat{Bed}{Bedingungen~(#2#1#3)}
\Crefmultiformat{Bed}{Bedingungen~(#2#1#3)}


%%%%%%%%%%%%%%%%%%%%%%%%%%%%%%%%%%%%%%%%%%%
%
%   Größere Seitenvorschau in Notizen
%
%%%%%%%%%%%%%%%%%%%%%%%%%%%%%%%%%%%%%%%%%%%
\newcommand{\VorschauHoehe}{0.50} %Prozent zur Höhe der gesamten Notizseite
\makeatletter
\defbeamertemplate{note page}{infolines}
{%
  {%
    \scriptsize
    \insertvrule{\VorschauHoehe \paperheight}{white!90!black}
    \vskip- \VorschauHoehe \paperheight
    \nointerlineskip
    \vbox{
      \hfill\insertslideintonotes{\VorschauHoehe}\hskip-\Gm@rmargin\hskip0pt%
      \vskip-\VorschauHoehe\paperheight%
      \nointerlineskip
      \begin{pgfpicture}{0cm}{0cm}{0cm}{0cm}
        \begin{pgflowlevelscope}{\pgftransformrotate{90}}
          {\pgftransformshift{\pgfpoint{-2cm}{0.2cm}}%
          \pgftext[base,left]{\footnotesize\the\year-\ifnum\month<10\relax0\fi\the\month-\ifnum\day<10\relax0\fi\the\day}}
        \end{pgflowlevelscope}
      \end{pgfpicture}}
    \nointerlineskip
    \vbox to \VorschauHoehe \paperheight{\vskip0.5em
      \hbox{\insertshorttitle[width=8cm]}%
      \setbox\beamer@tempbox=\hbox{\insertsection}%
      \hbox{\ifdim\wd\beamer@tempbox>1pt{\hskip4pt\raise3pt\hbox{\vrule
            width0.4pt height7pt\vrule width 9pt
            height0.4pt}}\hskip1pt\hbox{\begin{minipage}[t]{7.5cm}\def\breakhere{}\insertsection\end{minipage}}\fi%
      }%
      \setbox\beamer@tempbox=\hbox{\insertsubsection}%
      \hbox{\ifdim\wd\beamer@tempbox>1pt{\hskip17.4pt\raise3pt\hbox{\vrule
            width0.4pt height7pt\vrule width 9pt
            height0.4pt}}\hskip1pt\hbox{\begin{minipage}[t]{7.5cm}\def\breakhere{}\insertsubsection\end{minipage}}\fi%
      }%
      \setbox\beamer@tempbox=\hbox{\insertshortframetitle}%
      \hbox{\ifdim\wd\beamer@tempbox>1pt{\hskip30.8pt\raise3pt\hbox{\vrule
            width0.4pt height7pt\vrule width 9pt
            height0.4pt}}\hskip1pt\hbox{\insertshortframetitle[width=7cm]}\fi%
      }%
      \vfil}%
  }%
  \vskip.25em
  \nointerlineskip
  \begin{minipage}{\textwidth} % this is an addition
  \insertnote
  \end{minipage}               % this is an addition
}
\makeatother

\setbeamertemplate{note page}[infolines]
%%%

%Benutze
%\note{\trickdown \begin{enumerate}
%  \item< 1-> ...
%\end{enumerate}}

\makeatletter
    \newcommand{\trickdown}{%
        \advance\beamer@slideinframe by-1%
     }%

     \newcommand{\trickup}{%
         \advance\beamer@slideinframe by+1%
      }%
\makeatother


%!TeX spellcheck = de-DE

\DeclareMathOperator{\sgn}{sgn}
\DeclareMathOperator{\id}{id}
\DeclareMathOperator{\supp}{supp}
\DeclareMathOperator{\Vol}{Vol}
\DeclareMathOperator{\rot}{rot}
\DeclareMathOperator{\Int}{Int}
\DeclareMathOperator{\codim}{codim}
\DeclareMathOperator{\T}{T}
\DeclareMathOperator{\ind}{ind}
\DeclareMathOperator{\spn}{span}
\DeclareMathOperator{\dist}{dist}
\DeclareMathOperator{\GLp}{GL^+}
\DeclareMathOperator{\conv}{conv}
%\DeclareMathOperator{\ker}{ker}
\DeclareMathOperator{\img}{img}
\DeclareMathOperator{\rang}{rang}
\newcommand{\D}{\text{D}}

\newcommand{\QR}[2]{{ \left. \raisebox{0.2\height}{\ensuremath{#1}} \middle \diagup \raisebox{-0.2\height}{\ensuremath{#2}} \right. }}


% einige Abkuerzungen
\newcommand{\C}{\mathbb{C}} % komplexe
\newcommand{\K}{\mathbb{K}} % komplexe
\newcommand{\R}{\mathbb{R}} % reelle
\newcommand{\Q}{\mathbb{Q}} % rationale
\newcommand{\Z}{\mathbb{Z}} % ganze
\newcommand{\N}{\mathcal{N}} % Normalenraum

\NewDocumentCommand\set{d<>m}{
  \left\lbrace{} #2 %
  \IfValueTF{#1}{
    \;\middle|\; #1%
  }{
    %
  }
  \right\rbrace{}%
}




\makeatletter

\newcommand\frontmatter{%
    \cleardoublepage
  %\@mainmatterfalse
  \pagenumbering{roman}}

\newcommand\mainmatter{%
    \cleardoublepage
 % \@mainmattertrue
  \pagenumbering{arabic}}

\newcommand\backmatter{%
  \if@openright
    \cleardoublepage
  \else
    \clearpage
  \fi
 % \@mainmatterfalse
   }

\makeatother


% include config
\input{my_config}
%\makeatletter
%\let\th@plain\relax
%\makeatother

%\setbeamertemplate{theorems}[numbered] % to number
\setbeamertemplate{theorems}[ams style]

%%\usepackage{ntheorem} %nicht kompatibel mit uncover
\usepackage{amsthm}
%\theoremstyle{break}

%\newtheoremstyle{mybreak}%
%{0pt}{0pt}{\normalfont}%
%{}{\bfseries\color{rwthblue}}{}{\newline}%
%{}

\newtheoremstyle{mystyle}%                % Name
  {1pt}%                                     % Space above
  {3pt}%                                     % Space below
  {}%                                     % Body font
  {}%                                     % Indent amount
  {\color{rwthblue}}%                            % Theorem head font
  {}%                                    % Punctuation after theorem head
  {\newline}%                                    % Space after theorem head, ' ', or \newline
  {\thmname{#1}\thmnumber{ #2}\thmnote{ (#3)}}% 

\theoremstyle{mystyle}

%\theorembodyfont{\normalfont}
%\theoremheaderfont{\color{rwthblue}\bfseries}

\newtheorem*{Sa}{Theorem}
\newtheorem*{Def}{Definition}
\newtheorem*{Ko}{Korollar}
\newtheorem{Lem}{Lemma}
\newtheorem*{Bem}{Bemerkung}

%% um Beweise über mehrere Folien zu machen
\newtheorem*{bewfort}{Beweis (Forsetzung)}
\newtheorem*{bewanf}{Beweis}
\newenvironment{bewend}{\begin{bewfort}}{\hfill $\square$ \end{bewfort}}
\newenvironment{bew}{\begin{bewanf}}{\hfill $\square$ \end{bewanf}}
%%%%%

% disable PDF navigation icons
\setbeamertemplate{navigation symbols}{}
\pgfplotsset{compat=1.15}